\documentclass{beamer}

\usepackage{graphicx} % for \rotatebox

\usepackage{xeCJK}
\newfontlanguage{Chinese}{CHN}
\setCJKmainfont{SimSun}
\newfontfamily\songextb{PMingLiU-ExtB}	% goes with Simsun (Founder Extended)
\setCJKfamilyfont{songvert}[Script=CJK,Language=Chinese,Vertical=RotatedGlyphs]{SimSun}
\setromanfont[Scale=1,Mapping=tex-text]{TeX Gyre Heros Cn}

\newcommand*\CJKmovesymbol[1]{\raise.35em\hbox{#1}}
\newcommand*\CJKmove{\punctstyle{plain}% do not modify the spacing between punctuations
  \let\CJKsymbol\CJKmovesymbol
  \let\CJKpunctsymbol\CJKsymbol}
\def\CJKsymbol#1{%
  \iffontchar\font`#1%
    #1%
  \else
    {\songextb#1}%
  \fi}


\graphicspath{{./graphics/}}
\DeclareGraphicsExtensions{.jpg,.jpeg,.png}
\newcommand\gr[2]{\scalebox{#2}{\includegraphics{#1}}}

\title{CSc 221 Android Project Report}
\author{David Prager Branner}
\date{December 18, 2012\\\quad\\Grove School of Engineering\\City College of New York}

\begin{document}

\frame{\titlepage}


\frame{
  \frametitle{Basic idea}

  \begin{itemize}
  \item<2-> practice recognizing whole Chinese characters, to improve your speed (not vocabulary)
  \item<3-> characters chosen at random, one ``target'' assigned to user to find among ``decoys''
  \item<4-> app indicates right and wrong choices
  \end{itemize}
}



\frame{
 \frametitle{Demonstration}
}

\frame{
 \frametitle{Demonstration --- Opening screen, choose dimensions}
\begin{center}

\gr{05a1_open.png}{.25}
\end{center}
 }

\frame{
 \frametitle{Demonstration --- Shown target at bottom}
\begin{center}
\gr{05a2_choose.png}{.25}
\end{center}
 }

\frame{
 \frametitle{Demonstration --- Correct choice turns blue}
\begin{center}
\gr{05a3_chosen_right.png}{.25}
\end{center}
 }

\frame{
 \frametitle{Demonstration --- Incorrect choices turn red on black}
\begin{center}
\gr{05a4_chosen_wrong.png}{.25}
\end{center}
 }

\frame{
 \frametitle{Demonstration --- After incorrect choice, can choose again}
\begin{center}
\gr{05a5_chosen_wrong_then_right.png}{.25}
\end{center}
 }

\frame{
 \frametitle{Demonstration --- More interesting with more decoys}
\begin{center}
\gr{07a1_open.png}{.25}
\end{center}
 }

\frame{
 \frametitle{Demonstration --- More interesting with more decoys}
\begin{center}
\gr{07a2_choose.png}{.25}
\end{center}
 }

\frame{
 \frametitle{Demonstration --- \textbf{More} interesting with \textbf{more} decoys}
\begin{center}
\gr{16a1_open.png}{.25}
\end{center}
 }

\frame{
 \frametitle{Demonstration --- \textbf{More} interesting with \textbf{more} decoys}
\begin{center}
\gr{16a2_choose.png}{.25}
\end{center}
 }

\frame{
 \frametitle{Demonstration --- \textbf{More} interesting with \textbf{more} decoys}
\begin{center}
\gr{16a3_chosen_wrong.png}{.25}
\end{center}
 }

\frame{
 \frametitle{Demonstration --- \textbf{More} interesting with \textbf{more} decoys}
\begin{center}
\gr{16a4_chosen_wrong_then_right.png}{.25}
\end{center}
 }

\frame{
 \frametitle{Demonstration --- \textbf{\textit{More}} interesting with \textbf{\textit{more}} decoys!}
\begin{center}
\gr{20a1_open.png}{.25}
\end{center}
 }

\frame{
 \frametitle{Demonstration --- \textbf{\textit{More}} interesting with \textbf{\textit{more}} decoys!}
\begin{center}
\gr{20a2_choose.png}{.25}
\end{center}
 }

\frame{
 \frametitle{Demonstration --- \textbf{\textit{More}} interesting with \textbf{\textit{more}} decoys!}
\begin{center}
\gr{20a3_chosen_right.png}{.25}
\end{center}
 }

\frame{
 \frametitle{Demonstration --- OCD is helpful to studying Chinese}
\begin{center}
\gr{20a4_again.png}{.25}
\end{center}
 }

\frame{
 \frametitle{Demonstration --- OCD is helpful to studying Chinese}
\begin{center}
\gr{20a5_choose.png}{.25}
\end{center}
 }

\frame{
 \frametitle{Demonstration --- OCD is helpful to studying Chinese}
\begin{center}
\gr{20a6_again.png}{.25}
\end{center}
 }

\frame{
 \frametitle{Demonstration --- OCD is helpful to studying Chinese}
\begin{center}
\gr{20a7_choose.png}{.25}
\end{center}
 }

\frame{
 \frametitle{Demonstration --- More OCD is more helpful}
\begin{center}
\gr{20a8_again.png}{.25}
\end{center}
 }

\frame{
 \frametitle{Demonstration --- More OCD is more helpful}
\begin{center}
\gr{20a9_choose.png}{.25}
\end{center}
 }

\frame{
 \frametitle{Demonstration --- \textbf{More} OCD is \textbf{more} helpful}
\begin{center}
\gr{20a10_choose.png}{.25}
\end{center}
}

\frame{
 \frametitle{Implementation}
  \begin{itemize}
  \item<2-> Chinese characters generated as random Unicode values
  \item<3-> each placed in a single \texttt{TextView}, with some \texttt{OnClickListener} functionality
  \item<4-> each \texttt{TextView} is one cell in a \texttt{TableLayout}
  \item<5-> a secondary single-row \texttt{TableLayout} for navigation at screen bottom
  \item<6-> the two \texttt{TableLayout}s reside in a \texttt{RelativeLayout} (top level)
  \item<7-> character font-size optimized to fit screen
  \end{itemize}
}

\frame{
 \frametitle{Challenges}
 \begin{itemize}
 \item<2-> \textbf{choosing suitable layout}\uncover<3->{--- ``solved'' through trial and error}
 \item<4-> understanding how to actually \textbf{implement Google's API} \uncover<5->{--- no solution}\uncover<6->{; enormously time-consuming}\uncover<7->{; without StackOverflow.com there would be no hope}
 \item<8-> \textbf{dynamic formatting} of \texttt{TextViews} and dimensions of \texttt{TableLayout} \uncover<9->{--- minimal use of \texttt{activity\_main.xml} for layout}\uncover<10->{; mostly done programmatically}\uncover<11->{, which has led to other problems}
 \end{itemize}
}

\frame{
 \frametitle{Use of time}
 \begin{itemize}
 \item<2-> 26 days in all
 \item<3-> 1 spent on Chinese, writing proof of concept with plain Java and Swing
 \item<4-> 18 spent struggling with Google's ``Guide'' tutorials\uncover<5->{ (actual errors found)}
 \item<6-> 5 spent experimenting with layouts
 \item<7-> 2 spent trying to implement various methods and tuning performance
 \end{itemize}

 }

\frame{
 \frametitle{Unimplemented ideas}
 \begin{itemize}
 \item<2-> report time-score in \texttt{ActionBar}
 \item<3-> pedagogically more interesting choice of characters (right now, random)\uncover<4->{ --- requires database}
 \item<5-> obfuscation of screen to make identification more challenging
 \item<6-> proper user control of settings
 \item<7-> more game-like qualities
 \end{itemize}

 }

\frame{
 \frametitle{Problems}
 \begin{itemize}
 \item<2-> \texttt{NumberPicker} on opening screen does not pass value
 \item<3-> within \texttt{OnClickListener} environment, objects are not accessible\uncover<4->{ --- perhaps because of over-reliance on programmatic, non-XML layout}
 \item<5-> ``again'' button calls layout-building code recursively --- a recipe for disaster
 \end{itemize}

 }

\frame{
\begin{center}END

\huge{終}\end{center}
 }

%\frame{
% \frametitle{Topic of study: Parallel prose --- examples, parsed}
%
%\begin{table}[htdp]
%\begin{tabular}{cp{.6\textwidth}}
%
%\Large
%
%\rotatebox{-90}{\mbox{\begin{minipage}{10em}
%\CJKfamily{songvert}\CJKmove
%素愛其文\\
% ● ○\\
%不能釋手\\
% ○ ●
%\end{minipage}}}
%
%&
%
%\vskip12pt From Xiāo Tǒng 蕭統, ``\textit{Táo Yuānmíng jí} xù 陶淵明集序''
%
%\vskip12pt Traditional symbols: 
%
%\parindent2em○ \textit{píng}-tone syllable
%
%● \textit{zè} (non-\textit{píng}) syllable)\parindent0em
%\end{tabular}
%\end{table}%
%}
%
%\frame{
% \frametitle{Topic of study: Parallel prose --- examples, parsed}
%
%\begin{table}[htdp]
%\begin{tabular}{cp{.6\textwidth}}
%
%\Large
%
%\rotatebox{-90}{\mbox{\begin{minipage}{10em}
%\CJKfamily{songvert}\CJKmove
%素愛其文\\
% ● ○\\
%不能釋手\\
% ○ ●
%\end{minipage}}}
%
%&
%
%\vskip12pt From Xiāo Tǒng 蕭統, ``\textit{Táo Yuānmíng jí} xù 陶淵明集序''
%
%\vskip12pt Traditional symbols: 
%
%\parindent2em○ \textit{píng}-tone syllable
%
%● \textit{zè} (non-\textit{píng}) syllable)\parindent0em
%
%\vskip12pt This alternation of tone-types (within single lines and between related lines in a couplet) is very common from the mid-6th century onward.
%\end{tabular}
%\end{table}%
%}
%
%
%\frame{
% \frametitle{Topic of study: Parallel prose --- examples, parsed}
%
%\begin{table}[htdp]
%\begin{tabular}{cp{.6\textwidth}}
%
%\Large
%
%\rotatebox{-90}{\mbox{\begin{minipage}{10em}
%\CJKfamily{songvert}\CJKmove
%素愛其文\\
% ● ○\\
%不能釋手\\
% ○ ●
%\end{minipage}}}
%
%&
%
%
%\vskip12pt It is surely related to the aesthetics of performance. From the Míng we have records saying that \textit{píng} tones are to be prolonged and \textit{zè} tones left short (in contrast). This practice is easy to observe in modern performance in Taiwan and elsewhere.
%\end{tabular}
%\end{table}%
%}
%
%\frame{
% \frametitle{Topic of study: Parallel prose --- examples, parsed}
%
%\begin{table}[htdp]
%\begin{tabular}{cp{.6\textwidth}}
%
%\Large
%
%\rotatebox{-90}{\mbox{\begin{minipage}{10em}
%\CJKfamily{songvert}\CJKmove
%素愛其文\\
% ● ○\\
%不能釋手\\
% ○ ●
%\end{minipage}}}
%
%&
%
%\vskip12pt The retrospective Chinese name of this style, from the Manchu ero, has been \textit{piánwén} 駢文 'teamed-[horse] writing'.
%
%\vskip12pt In English, often termed ``parallelism'' or ``antithesis''. Unsatisfactory.
%
%\vskip12pt Chinese \textit{duì} 對 n. 'couplet', v. 'to pair, to match, to put in opposition to'.
%
%\vskip12pt Chinese \textit{duìchèn} 對稱/對襯 `symmetrical'. Lit. ``\textbf{counter-balanced}''.
%\end{tabular}
%\end{table}%
%}
%
%\title{Are There Regional Styles of Prosody\\in Sixth-Century \textbf{Counterbalanced} Prose?}
%\author{David Prager Branner, with William E. Skeith, III\\``Poetry and Place: The Rise of the South''\\Princeton University}
%\date{October 27, 2012}
%
%\frame{\titlepage}
%
%\frame{
% \frametitle{Topic of study: Parallel prose}
% }
%
%
%\frame{
% \frametitle{Topic of study: Parallel prose --- more complex cases}
%
%\begin{table}[htdp]
%\begin{tabular}{cp{.4\textwidth}}
%
%\Large
%
%\rotatebox{-90}{\mbox{\begin{minipage}{10em}
%\CJKfamily{songvert}\CJKmove
%西蜀豪家\\
%\\
%託情窮於魯殿\\
%\\
%東儲甲觀\\
%\\
%誦詠止於洞簫\\
% 
%\end{minipage}}}
%
%&
%
%\vskip12pt From Xú Líng 徐陵, ``\textit{Yùtái xīnyǒng} xù 玉臺新詠序''.
%
%\end{tabular}
%\end{table}%
%}
%
%
%\frame{
% \frametitle{Topic of study: Parallel prose --- more complex cases}
%
%\begin{table}[htdp]
%\begin{tabular}{cp{.4\textwidth}}
%
%\Large
%
%\rotatebox{-90}{\mbox{\begin{minipage}{10em}
%\CJKfamily{songvert}\CJKmove
%西蜀豪家\\
%\\
%託情窮\textbf{於}魯殿\\
%\\
%東儲甲觀\\
%\\
%誦詠止\textbf{於}洞簫\\
% 
%\end{minipage}}}
%
%&
%
%\vskip12pt From Xú Líng 徐陵, ``\textit{Yùtái xīnyǒng} xù 玉臺新詠序''.
%
%\vskip12pt Example of ``interlarded'' couplets (\textit{géjù duì} 隔句對). From line-lengths and semantics alone this is evident.
%
%\end{tabular}
%\end{table}%
%}
%
%
%\frame{
% \frametitle{Topic of study: Parallel prose --- more complex cases}
%
%\begin{table}[htdp]
%\begin{tabular}{cp{.4\textwidth}}
%
%\Large
%
%\rotatebox{-90}{\mbox{\begin{minipage}{10em}
%\CJKfamily{songvert}\CJKmove
%西蜀豪家\\
%\\
%託情窮於魯殿\\
%\\
%東儲甲觀\\
%\\
%誦詠止\textbf{於}洞簫\\
% 
%\end{minipage}}}
%
%&
%
%Someone from a powerful clan, of Western Shǔ,
%
%\vskip6pt gave expression to his feelings, but to exhaustion, \textbf{with} ``Lǔ Palace'' ---
%
%\vskip6pt while a Crown Prince, in his First Temple,
%
%\vskip6pt chanted, but stopped \textbf{at} ``The holed flute.'' \parindent0em
%
%\end{tabular}
%\end{table}%
%}



\end{document}
